%% PNAStwoS.tex
%% Sample file to use for PNAS articles prepared in LaTeX
%% For two column PNAS articles
%% Version1: Apr 15, 2008
%% Version2: Oct 04, 2013



%% BASIC CLASS FILE
\documentclass{pnastwo}






% And another fix.  PNAS class loses the label of floats unless they       
% were defined with the [h] option (so not really floats at all).  It      
% all comes down to wrong scope in the following routine which pushes      
% out the floats onto the page.  This is the fixed version:        
\makeatletter                                  
\def\DonormalEndcol{%                              
%% top float ==>                               
\ifx\toporbotfloat\xtopfloat%                          
%% figure ==>                                  
  \ifcaptypefig%                               
  \expandafter\gdef\csname topfloat\the\figandtabnumber\endcsname{%    
  \vbox{\vskip\PushOneColTopFig%                       
  \unvbox\csname figandtabbox\the\loopnum\endcsname%               
  \vskip\abovefigcaptionskip%                          
  \csname caption\the\loopnum\endcsname%                   
  \csname letteredcaption\the\loopnum\endcsname%               
  \csname continuedcaption\the\loopnum\endcsname%              
  \csname letteredcontcaption\the\loopnum\endcsname            
  \ifredefining%                               
  \csname label\the\loopnum\endcsname%                     
  \expandafter\gdef\csname topfloat\the\loopnum\endcsname{}\fi}%       
  \vskip\intextfloatskip%%                         
  \vskip-4pt %% probably an artifact of topskip??              
}%                                     
\else%                                     
%% plate ==>                                   
  \ifcaptypeplate%                             
  \expandafter\gdef\csname topfloat\the\figandtabnumber\endcsname{%    
  \vbox{\vskip\PushOneColTopFig%                       
  \unvbox\csname figandtabbox\the\loopnum\endcsname            
  \vskip\abovefigcaptionskip                           
  \csname caption\the\loopnum\endcsname                    
  \csname letteredcaption\the\loopnum\endcsname                
  \csname continuedcaption\the\loopnum\endcsname               
  \csname letteredcontcaption\the\loopnum\endcsname            
  \ifredefining                                
  \csname label\the\loopnum\endcsname                      
  \expandafter\gdef\csname topfloat\the\loopnum\endcsname{}\fi}        
  \vskip\intextfloatskip %%                            
  \vskip-4pt %% probably an artifact of topskip??              
}%                                     
\else% table ==>                               
 \expandafter\gdef\csname topfloat\the\figandtabnumber\endcsname{%     
 \vbox{\vskip\PushOneColTopTab %%                      
 \csname caption\the\loopnum\endcsname                     
  \csname letteredcaption\the\loopnum\endcsname                
  \csname continuedcaption\the\loopnum\endcsname               
  \csname letteredcontcaption\the\loopnum\endcsname            
  \vskip\captionskip                               
  \unvbox\csname figandtabbox\the\loopnum\endcsname            
\ifredefining                                  
\csname label\the\loopnum\endcsname                    
\expandafter\gdef\csname topfloat\the\loopnum\endcsname{}\fi           
}\vskip\intextfloatskip %% why don't we need this?             
\vskip-10pt}                                   
\fi\fi%                                    
%                                      
\else% bottom float                            
%                                      
\ifcaptypefig                                  
\expandafter\gdef\csname botfloat\the\figandtabnumber\endcsname{%      
\vskip\intextfloatskip                             
\vbox{\unvbox\csname figandtabbox\the\loopnum\endcsname            
\vskip\abovefigcaptionskip                         
  \csname caption\the\loopnum\endcsname                    
  \csname letteredcaption\the\loopnum\endcsname%               
  \csname continuedcaption\the\loopnum\endcsname%              
  \csname letteredcontcaption\the\loopnum\endcsname%               
\vskip\PushOneColBotFig%%                          
\ifredefining%                                 
\csname label\the\loopnum\endcsname                    
\expandafter\gdef\csname botfloat\the\loopnum\endcsname{}\fi}}%        
\else                                      
\ifcaptypeplate                                
\expandafter\gdef\csname botfloat\the\figandtabnumber\endcsname{%      
\vskip\intextfloatskip                             
\vbox{\unvbox\csname figandtabbox\the\loopnum\endcsname            
\vskip\abovefigcaptionskip                         
  \csname caption\the\loopnum\endcsname                    
  \csname letteredcaption\the\loopnum\endcsname%               
  \csname continuedcaption\the\loopnum\endcsname%              
  \csname letteredcontcaption\the\loopnum\endcsname%               
\vskip\PushOneColBotFig%%                          
\ifredefining%                                 
\csname label\the\loopnum\endcsname                    
\expandafter\gdef\csname botfloat\the\loopnum\endcsname{}\fi}}%        
  \else% TABLE                                 
\expandafter\gdef\csname botfloat\the\figandtabnumber\endcsname{%      
  \vskip\intextfloatskip                           
\vbox{\csname caption\the\loopnum\endcsname                
  \csname letteredcaption\the\loopnum\endcsname                
  \csname continuedcaption\the\loopnum\endcsname               
  \csname letteredcontcaption\the\loopnum\endcsname%               
  \vskip.5\intextfloatskip                         
  \unvbox\csname figandtabbox\the\loopnum\endcsname%               
\vskip\PushOneColBotTab                            
\ifredefining%                                 
\csname label\the\loopnum\endcsname                    
\expandafter\gdef\csname botfloat\the\loopnum\endcsname{}\fi}}%        
\fi\fi\fi}                                 
\makeatother  


%Fix wierd behavior which prevents table captions from appearing for
% tables in the body of the article
\makeatletter
\long\def\@makecaption#1#2{%
\ifx\@captype\table
\let\currtabcaption\relax
\gdef\currtabcaption{
\tabnumfont\relax #1. \tabtextfont\relax#2\par
\vskip\belowcaptionskip 
}
\else
 \vskip\abovecaptionskip
  \sbox\@tempboxa{\fignumfont#1.\figtextfont\hskip.5em\relax #2}%
  \ifdim \wd\@tempboxa >\hsize
\fignumfont\relax #1.\figtextfont\hskip.5em\relax#2\par
  \else
    \global \@minipagefalse
    \hb@xt@\hsize{\hfil\box\@tempboxa\hfil}%
  \fi
\fi
}
\makeatother


\usepackage{adjustbox}
\usepackage{array}

% \newcolumntype{R}[2]{%
%     >{\adjustbox{angle=#1,lap=\width-(#2)}\bgroup}%
%     l%
%     <{\egroup}%
% }
\newcolumntype{R}[2]{%
    >{\adjustbox{angle=#1,lap=\width-(#2)}\bgroup}%
    l%
    <{\egroup}%
}
\newcommand*\rot{\multicolumn{1}{R{75}{0em}}}% no optional argument here, please!


\setlength{\footskip}{.5in}
\usepackage{algorithm2e}
%% ADDITIONAL OPTIONAL STYLE FILES Font specification
\usepackage{natbib}
\usepackage{bm}% bold math
%\newcommand{\bm}[1]{\boldsymbol{#1}} %makes bold math symbols easier
\newcommand{\R}{\textsf{R}\space} %R in textsf font
\newcommand{\X}{\bm{\mathcal{X}}} %shorthand for iid
\renewcommand{\P}{\mathcal{P}}
\newcommand{\bt}{\pmb{\theta}}
\newcommand{\bl}{\pmb{\lambda}}
\newcommand{\bL}{\pmb{\Lambda}}
%\newcommand{\bG}{\pmb{\Gamma}}
\newcommand{\bh}{\pmb{\text{h}}}
\newcommand{\h}{\pmb{\text{h}}}
\usepackage{amsmath,amssymb,amsthm}
\def\citeapos#1{\citeauthor{#1}'s (\citeyear{#1})}
\DeclareMathOperator*{\argmax}{arg\,max}

%\graphicspath{{/Users/matthewjdenny/Dropbox/PINLab/Projects/Denny_Working_Directory/2011_Analysis_Output}}

%\usepackage{pnastwoF}
\usepackage{hyperref,booktabs,tabularx,float}


%% OPTIONAL MACRO DEFINITIONS
\def\s{\sigma}


\begin{document}

\title{Reading between the Emails: Gendered Patterns of Communication in Local Government}

\author{
Matthew Denny\affil{1}{Penn State University},
James ben-Aaron\affil{2}{University of Massachusetts Amherst},
Hanna Wallach\affil{2}{}\affil{3}{Microsoft Research NYC},
\and Bruce Desmarais\affil{1}{}
}

\contributor{\vspace{-.25cm}}


\maketitle

\begin{article}
\begin{abstract}
{We conduct an analysis of the relationship between gender and communication patterns in a sample of 17 North Carolina county governments. We apply an extension of the recently developed topic-partitioned multinetwork embeddings model to infer the content-conditional structure of manager-to-manager email communication in these county governments. We provide results which illustrate that while aggregate patterns of communication among department managers do not display significant gender bias, the content of communication and positions held by men and women differ significantly. We also find that some previously studied institutional-level factors seem to matter for the gendering of communication, and that women seem to be excluded from the locus of control in these organizations.
%We illustrate the use of covariates in the model with gender and find that patterns of gender mixing vary with the topical content of communication in a way that is consistent with the content-conditional gendered patterns that have been found to characterize social and organizational networks
}
\end{abstract} 

%\keywords{weighted networks |  | x-ray reflectivity | molecular electronics}

%\abbreviations{SAM, self-assembled monolayer; OTS, octadecyltrichlorosilane}

\section{Gender in Organizations}
Gender bias against female members of firms and government organizations is well documented in terms of pay, prestige, opportunity for advancement, and social interaction \citep{Brass1985, Bielby1986a, Ibarra1992, Albrecht2003, Duncan2004}. At the same time, providing equal treatment to male and female colleagues is a key objective for organizations that aspire to a just, efficient, and sustainable professional culture \citep{Ely2000}. Scholars have sought to understand the roots, extent, and nature of gender bias, but have only had limited access to primary-source data in previous studies. The data used in these studies \citep[e.g.,][]{Castilla2005, Adams2007, Elsesser2011} has traditionally been observational, ethnographic or consists of self reports, which can be biased and are often limited in scope. 
	
However, with the increasing use of electronic communication, and the rise of \emph{e-government} and transparency initiatives within government, scholars are now able to use public records requests to gather primary source email communication data. Using public records requests as a method for data collection also carries a major benefit of making the data collection process replicable. Therefore, we choose to study the relationship between gender and communication patterns in a sample of local governments. To do this, we collect and analyze a large scale email data set from a sample of seventeen county governments in North Carolina. We look for gendered patterns of communication at multiple overlapping scales within and across these organizations, which we find evidence for, using both descriptive statistics and a new statistical model for email data. Our results indicate that the relationship between gender and the patterns of communication in these organizations is complex, multi-scale, and strongly related to the positions men and women hold within these organizations.


\section{Data}

The data used in this study were collected in 2013 via a series of public records requests made to all 100 North Carolina county governments. We selected North Carolina for our data collection efforts because the state public records laws cover email data, and prevent counties from charging unreasonable fees for providing data. 23 counties complied with our request, of which 17 provided data was useful to our research\footnote{Some counties provided too little data, or provided data in paper form, which was not amenable to our analyses.}. Our records request to each county included all emails sent and received by all department managers (e.g. Health, Finance, etc.) over overlapping 3 month periods in 2013. We received approximately 500,000 total emails from the 17 participating county governments, of which approximately 17,000 were sent between county department managers. Figure \ref{fig:nc map} highlights the counties which responded to our requests for data. We also collected detailed metadata on the department and gender of the 362 department managers in our sample. We display some basic descriptive statistics for each county in table \ref{tab:county aggregate stats}. We can see from table \ref{tab:county aggregate stats} that approximately 40\% of department managers in our sample are women, although there is significant variation in the proportion of female department managers across counties. There is also significant variation in the number of emails provided by these counties. 


	\begin{figure}
		\centering
	\caption{\label{fig:nc map} North Carolina county map with counties that provided data highlighted.}	
	\centering
	\includegraphics[width = 0.48\textwidth]{images/County_Map.pdf}
	\end{figure}
	
	\begin{table}
	\centering
		\begin{tabular}{lrrrr}
		  \hline
		  % add an extra line
		  & \multicolumn{2}{c}{\textbf{Manager Gender}} & \\
		  \cmidrule{2-3}
		 \textbf{County} & \textbf{Male} & \textbf{Female} & \textbf{Emails Sent}  \\
		  \hline
		Alexander & 12 & 9 & 907   \\
		Caldwell & 12 & 8 & 121     \\
		Chowan & 12 & 11 & 2,027   \\
		Columbus & 14 & 10 & 920   \\
		Dare & 15 & 12 & 2,247    \\
		Duplin & 13 & 14 & 1,914    \\
		Hoke & 13 & 11 & 1,106  \\
		Jackson & 18 & 6 & 1,499    \\
		Lenoir & 15 & 5 & 560  \\
		Lincoln & 15 & 7 & 573   \\
		McDowell & 12 & 5 & 326   \\
		Montgomery & 8 & 10 & 680   \\
		Nash & 11 & 8 & 1,147  \\
		Person & 12 & 9 & 1,491   \\
		Transylvania & 16 & 4 & 1,857  \\
		Vance & 10 & 8 & 185   \\
		Wilkes & 15 & 2 & 303   \\
		   \hline
		   \textbf{Totals:} & 362 & 139 & 17,863 \\
		   \hline
		\end{tabular}
		\caption{\label{tab:county aggregate stats}Department manager gender breakdown and number of emails sent by department managers for each county in our sample. It is important to note that the counties in our sample are statistically indistinguishable from the rest of the counties in North Carolina on a number of demographic dimensions including population, per-capita income, and percent of the population that is white. \\}
	\end{table}
	
	% \begin{figure}
	% 	\centering
	% \caption{\label{fig:nc map} North Carolina county map.}
	% \centering
	% \includegraphics[width = 0.48\textwidth]{images/County_Map.pdf}
	% \end{figure}
	%
	% \begin{table}
	% \centering
	% 	\begin{tabular}{lrrrrr}
	% 	  \hline
	% 	  % add an extra line
	% 	  & \multicolumn{2}{c}{\textbf{Manager Gender}} & \multicolumn{2}{c}{\textbf{Email Sender}}\\
	% 	 \textbf{County} & \textbf{Male} & \textbf{Female} & \textbf{Manager} &\textbf{All} \\
	% 	  \hline
	% 	Alexander & 12 & 9 & 907 & 11,924  \\
	% 	Caldwell & 12 & 8 & 121 &    \\
	% 	Chowan & 12 & 11 & 2,027 & 11,737  \\
	% 	Columbus & 14 & 10 & 920 & 12,707  \\
	% 	Dare & 15 & 12 & 2,247 &   \\
	% 	Duplin & 13 & 14 & 1,914 &   \\
	% 	Hoke & 13 & 11 & 1,106 & 5,565  \\
	% 	Jackson & 18 & 6 & 1,499 &   \\
	% 	Lenoir & 15 & 5 & 560 & 10,499 \\
	% 	Lincoln & 15 & 7 & 573 & 8,727  \\
	% 	McDowell & 12 & 5 & 326 & 3,494  \\
	% 	Montgomery & 8 & 10 & 680 & 2,465  \\
	% 	Nash & 11 & 8 & 1,147 & 9,133 \\
	% 	Person & 12 & 9 & 1,491 & 14023  \\
	% 	Transylvania & 16 & 4 & 1,857 & 14,088 \\
	% 	Vance & 10 & 8 & 185 & 4,349  \\
	% 	Wilkes & 15 & 2 & 303 & 8,443  \\
	% 	   \hline
	% 	   \textbf{Totals:} & 362 & 139 & 17,863 & 117,154  \\
	% 	   \hline
	% 	\end{tabular}
	% 	\caption{\label{tab:county aggregate stats}Participating county email statistics. Note that in this study, we only make use of the internal (manager to manager) email data. Some email \textbf{All}'s are omitted due to challenges in determining which emails (not sent by managers) were valid in these counties.\\}
	% \end{table}


\section{Descriptive Analysis}

We begin our analysis of the relationship between gender and communication patterns in these county governments by looking for differences in aggregate email sending and receiving behavior, by gender, across our entire sample. Table \ref{tab:email agg stats} provides some basic descriptive statistics of the average number emails sent and received by male and female department managers in our sample. We can see that on average, male and female department managers send and receive a comparable number of emails, and that each email sent by male and female department managers has a similar number of recipients on average. These statistics tell us that if we observe gender differences in email communication, they are unlikely driven by some innate difference in the propensity for male and female department managers to send or receive emails. 
	
	
	\begin{table}
	\centering
		\begin{tabular}{m{2in}rrr}
		\toprule
		& \multicolumn{2}{c}{\textbf{Manager Gender}} \\
		\cmidrule{2-3}
	& \textbf{Male} & \textbf{Female}  \\
		 \midrule
		 % Proportion of Managers in Sample & 61.6\%& 38.4\% \\
		 % \midrule
		 Average number of emails sent & 48.3 & 51 \\
		 Average number of recipients per email sent & 1.45 & 1.43 \\
		 \midrule
		 Average number of emails received & 70.8 & 71.6 \\
		\bottomrule
		\end{tabular}
		\caption{\label{tab:email agg stats}Email statistics by manager
gender.\\}
	\end{table}
	
We can now disaggregate email sending and receiving by gender to see if there is a relationship between the gender of the sender of an email and the gender of the recipients of that email. If we do see a relationship, this would indicate that there is gender bias in communication within these organizations. To test whether the gender of an email sender is related to the gender of its recipients, in aggregate, we construct a contingency table of the gender of the sender of an email against the gender of the recipients of that email (see table \ref{tab:gender email agg stats}). We then perform a $\chi^2$ test for independence between the rows and columns.
	
	% make this a contingency table and do a chi squared test
	\begin{table}
	\centering
		\begin{tabular}{rlrr|r}
		\toprule
		 && \multicolumn{3}{c}{\textbf{Recipient}} \\
		\cmidrule{3-5}
	& & \textbf{Male} & \textbf{Female} & \textbf{Total}  \\
		 \midrule
		$|$& \textbf{Male} & 7,299 & 6,286 & 13,585 \\
	\textbf{Sender}	$|$& \textbf{Female} & 5,325 & 3,510 & 8,835 \\
	\cmidrule{2-5}
		 $|$& \textbf{Total} & 12,624 & 9,796 & \\
		\bottomrule
		\end{tabular}
		\caption{\label{tab:gender email agg stats}Each cell records the number of times a department manager of gender X was included as a recipient of an email sent by a department manager of gender Y. Statistics provided are calculated for all counties combined. Note that each email may have more than one recipient.}
	\end{table}
	
We conduct a $\chi^2$ test on this contingency table and the test statistic is $\chi^2 = 92.9$ with a p-value $< 0.00001$, indicating that the gender of an email sender and its recipients is not independent. While this result provides evidence for gender bias in communication in these county governments, our findings could potentially be confounded by gender differences in the departments men and women manage. Fortunately, our data includes the department managed by each manager, allowing us to disaggregate our analysis by department. We begin by looking for aggregate differences in the gender composition of departments (see table \ref{tab:gender position}). Departments were hand coded into one of 25 different categories based on the title given the county directory, to group departments that perform a similar function. For example the  
	
	
	% latex table generated in R 3.2.2 by xtable 1.7-4 package
	% Sat Sep 19 21:03:09 2015
	\setlength{\tabcolsep}{4pt}
	\begin{table}
	\centering
	         
	\begin{tabular}{rrrrrrrrrrrrrr}
	 & \rot{\textbf{Emergency}} & \rot{\textbf{Health}} & \rot{\textbf{IT}} & \rot{\textbf{Manager}} & \rot{\textbf{HR}} & \rot{\textbf{Library}} & \rot{\textbf{Plan/Dev}} & \rot{\textbf{Deeds}} & \rot{\textbf{Parks/Rec}} & \rot{\textbf{Finance}} & \rot{\textbf{Soc\_Serv}} & \rot{\textbf{Veterans}} & \rot{\textbf{Util/Waste}}  \\ 
	  \hline
	  \textbf{Male} & 15 & 5 & 11 & 15 & 3 & 3 & 11 & 6 & 9 & 5 & 8 & 5 & 11  \\ 
	\textbf{Female} & 2 & 11 & 2 & 2 & 12 & 8 & 3 & 9 & 5 & 12 & 8 & 7 & 1  \\ 
	  
	   \hline
	   & \rot{\textbf{Elections}} & \rot{\textbf{Sheriff}} & \rot{\textbf{Info}} & \rot{\textbf{Tax}} & \rot{\textbf{Inspections}} & \rot{\textbf{Animal}} & \rot{\textbf{Maintenance}} & \rot{\textbf{Seniors}} & \rot{\textbf{Transport}} & \rot{\textbf{Environment}} & \rot{\textbf{Misc}} & \rot{\textbf{Extension}} \\
	   \hline
	   \textbf{Male} &  2 & 16 & 2 & 10 & 11 & 9 & 5 & 2 & 6 & 7 & 5 & 8 \\
	   \textbf{Female} & 11 & 1 & 5 & 5 & 3 & 3 & 0 & 6 & 1 & 4 & 1 & 5 \\
	   \hline
	\end{tabular}
	\caption{\label{tab:gender position} Number of male and female managers for each department. Note that not all departments are represented in each county.\\}
	\end{table}
	\setlength{\tabcolsep}{6pt}
	
	
We can see that the managers of some department are mostly men: the county manager (the supervisor of all other department managers), the sheriff, and the emergency manager, for example. Other departments  are mostly managed by women -- the HR, finance, and health departments, for example. A a $\chi^2$ test for the independence of department and gender confirms our qualitative finding that there is a strong bias in which departments tend to be managed by women $(\chi^2 = 130.8, p < 0.00001)$. These results indicate a bias in the gender composition of different departments, but cannot tell us about differences in email communication by men and women who manage the same department. To answer this question, we want to know whether the gender of an email sender and the gender of its recipients is independent of the their respective departments. 
	
	\begin{figure*}
	\centering
	\includegraphics[width = 0.8\textwidth]{images/Aggregate_Email_Flows.pdf}
	\caption{\label{fig:heatmaps}Heat map depicting the total number of emails sent from the managers of each department to the managers of each other department, aggregated across counties. The right-hand x-axis displays the number of counties that had that department, and the number of those managers who were women. Departments are ordered by the total number emails their managers sent.}
	\end{figure*}
	% use recipeint rather than receiver and flip the y axis. omit zeros. have the shading march along together
	
	
Table \ref{fig:heatmaps} displays the aggregate number of times each department manager was the recipient of an email from the manager of each other department. We see pronounced differences in email sending an receiving by department, but cannot graphically disentangle the role of gender. To test for the independence of the gender of an email sender and the gender of its recipients from which department they direct, we construct a contingency table of department dyad-types (where each ``dyad-type'' is a unique combination of sender and recipient department, eg. finance $\longrightarrow$ HR) against gender dyad-types (where each ``dyad-type'' is a unique combination of sender and recipient gender, eg. male $\longrightarrow$ female).  We then perform a $\chi^2$ test for independence between the rows and columns. The test statistic $\chi^2 = 37,404$ with a simulated p-value\footnote{We use bootstrapped p values calculated using 50,000 resamples as they are more conservative.} $< 0.00002$, indicates that the gender of an email sender and its recipients is not independent of the their respective departments.
	
This result indicates that there is gender bias in communication when we disaggregate to the department-dyad level, however, we are still stuck. This analysis does not allow us to disentangle the two potential sources of the pattern we find -- bias in the positions women hold within these organizations, and bias in the way that male and female managers communicate. Furthermore, this analysis may be slicing our data too thin since not all counties have the same departments, and any particular department dyad may only exchange a handful of emails. Moreover, the department attribute of each manager is likely caused by several factors other than the topics of communication which may relate to gender bias. 
	
One solution to the problems raised above is to model the email content, as general topics such as balancing budgets are commonly represented across counties. Practically, this also allows us to focus on a smaller number of communication topics that are shared across counties.
	
To do this, we might want to use a statistical topic model to categorize emails, then model  gender mixing using the LSM in each topic.
	
However, the gender mixing parameters we infer using this approach would be confounded by selection effects. As a toy example, we would not be able to tell whether men prefer to talk to male coworkers over female coworkers about football, or women prefer not to talk about football at work, and so they do not participate in those conversations. 
	
To disentangle these effects, we need a joint model for email topical content, and the structure of communication.
	


\section{A Model of Email Content}
Here we provide a brief overview of our extension to the TPME model \citep{Krafft2012}. We focus on the ways in which our new model allows us to make inferences about the gender-specific patterns of communication in organizations, and how these patterns vary with the content of communication. Our model follows \citep{Krafft2012} by integrating the latent space network model \citep{Hoff2002a} with a statistical topic model (latent Dirichlet allocation \cite{Blei2003}) to jointly model the content and structure of a communication network. 

At a high level, our model assumes the following generative process for a message sent across the communication network. First, we sample the content of the message following the generative process for latent Dirichlet allocation. Then, for a given message sender, we sample whether each other actor in the network is a receiver of that message following the latent space network model. However, different topics (or clusters of topics in our extension) are associated with different latent spaces, so how likely each actor is to be a receiver of a particular message is dependent on the topical content of that message. We review this process in greater detail below.  

 \begin{figure}
\caption{\label{fig:splitting} Different patterns of communication across different domains.}	
\centering
\includegraphics[width = 0.48\textwidth]{images/Structure_Matters_Full.pdf}
\end{figure}

The message content is assumed to be generated via latent Dirichlet allocation. Under this model, we assume that all unique words in our vocabulary are associated to varying degrees with each of $T$ latent ``topics''. Each topic is then a distribution over unique words (or ``word types''), $\boldsymbol{\phi}^{(t)}$. For example ``doctor'', ``virus'', and ``medicine'' might all have high probability in a topic about hospitals, while ``cat'' and ``music'' might have low probability in that topic. We further assume that each topic distribution is drawn from a Dirichlet prior: 
\begin{equation}
	\boldsymbol{\phi}^{(t)} \sim \text{Dirichlet}(\beta,\boldsymbol{n}) \text{ for } t = \{1, ..., T\}
\end{equation}
The concentration parameter $\beta$ controls how peaky this distribution is (how few word types will have high probability in a given topic), and the base measure $\boldsymbol{n}$ controls the bias in the degree to which word types will be assigned high probability in topics generally. For example, ``patient'' might generally have a higher probability than most word types in a corpus of documents about healthcare.   

To generate each of the $D$ documents (messages) in our corpus, we first draw a document-specific distribution over topics
\begin{equation}
	\boldsymbol{\theta}^{(d)} \sim \text{Dirichlet}(\alpha,\boldsymbol{m}) \text{ for } d = \{1,...,D\}
\end{equation}
Here, the concentration parameter $\alpha$ controls how topically specific the documents generated are, and $\boldsymbol{m}$ controls the bias in how likely certain topics are to appear. Once we have sampled our document-specific distribution over topics we can then generate each of $N$ tokens (words) in the document via a two step process. First we draw the latent topic assignment for that token from the document specific distribution over topics.
\begin{equation}
	z_n^{(d)} \sim \boldsymbol{\theta}^{(d)} \text{ for } n = \{1,...,N\}
\end{equation}
 Then we draw the word type of that token from the topic specific distribution over word types.
 \begin{equation}
 	w_n^{(d)} \sim \boldsymbol{\phi}^{(z_n^{(d)})} \text{ for } n = \{1,...,N\}
 \end{equation}
We proceed in this manner for all tokens in all documents. 

% The content of each message sent across the network is assumed to be generated similarly to the generative process for latent Dirichlet allocation, where each message is a mixture of different ``topics'', but the mixture of topics, and thus words, is governed in part by the (intended) recipients of that message.  The sender then selects the recipients for that message probabilistically, based on the message content and the associated content-specific latent social structure of the network.

In our extension of TPME, we further assume each topic is uniquely associated with one of $C$ clusters, sampled from a discrete uniform distribution.
\begin{equation}
	C_t \sim \text{Discrete Uniform}(1,C) \text{ for } t = \{1, ..., T\}
\end{equation}
The intuition is that different broad content areas of communication (e.g. ``planning'', ``sports'',``meeting planning'') -- each of which is a collection of more specific topics -- will imply a different pattern of communication. Importantly, under this model, a message can be about a number of topics, and these topics can be associated with different clusters. Each cluster is associated with a different pattern of communication, so the receivers for any particular message must be sampled following an add-mixture of several underlying communication patterns. For example, if a department manager in an organization sent an email message to schedule a budget meeting, they would likely include both staff whose job includes setting up meetings, and staff who needed to provide input on the budget as recipients. 

Finally, for a given cluster $c$, the probability that an actor is selected as a recipient of a particular message is specified by the latent space network model \citep{Hoff2002a}. Under this model, we assume there is some baseline propensity to include message recipients on any email, which is governed by an intercept parameter.
\begin{equation}
	b^{c} \sim \text{Normal}(\mu, \tau^2)
\end{equation} 
For example, messages about sensitive HR matters will probably include fewer recipients than messages announcing a department party. Second, some attributes of the sender and potential receiver (which we assume are observed) may also affect the probability of that actor being a recipient. The effect that each of these $L$ different attributes have on the probability of an actor being included as a message recipient can then be parameterized by a vector: 
\begin{equation}
	\mathbf{\gamma}_l^{c} \sim \text{M. V. Normal}(\mathbf{\lambda}, \mathbf{\eta}^2) \text{ for } l = \{1, ..., L\}
\end{equation}
For example, there is a great deal of social science literature showing that people tend to preferentially communicate with others of the same gender. Our model could generate this propensity by sampling positive parameters for Male-Male and Female-Female communication, and negative parameters for Female-Male and Male-Female communication. 

The additional variation (not captured by covariate effects) in the probability that a message is sent between two actors is governed by how close they are in a $k$ dimensional ``latent social space''. This latent space captures all un-modeled factors associated with the propensity for two actors to form a tie. In a typical social network this might include difficult or impossible to measure quantities like how nice a person is, or whether two people have ``chemistry''. Additionally, it may capture observable traits that the researcher was unable to collect data on (e.g. sexual orientation, or ethnicity). This makes the latent positions difficult to interpret when including covariates in the model, so great care should be taken when doing so. To capture these latent positions, each actor $a$ is assumed to have some position in the $k$-dimensional latent space
\begin{equation}
	\mathbf{s}_a^{c} \sim \text{M. V. Normal}(0, \sigma^2) \text{ for } a = \{1, ...,A\}
\end{equation}
and the probability that they send a message to an actor $r \neq a$ is decreasing in their latent distance from $r$. Thus the probability of actor $r$ being a recipient of a message from actor $a$ under this model is:
\begin{equation}
	P(y_{a,r}^{c} = 1) = \text{logit}^{-1}\left( b^c + \sum_L \left[\mathbf{X}_{a,r}^{l} \mathbf{\gamma}_l^{c}\right] - |\mathbf{s}_a^{c} - \mathbf{s}_r^{c}| \right)
\end{equation}
where $\mathbf{X}^{l}$ is a matrix recording the type of an edge for attribute $l$ that would be sent between $a$ and $r$ (e.g. Male to Female or employee to supervisor). Edge values for an individual message are then sampled via Bernoulli trials. However, the probability that $y_{a,r} = 1$ for a given message $d$ may be dependent on multiple latent spaces (because documents can be about multiple topics). The weight given to $P(y_{a,r}^{(c)})$ for each cluster is determined by the proportion of tokens in the message that are assigned to topics associated with cluster $c$.  
\begin{equation}
	P(y_{a,r}^{(d)} = 1) = \sum_C P(y_{a,r}^{c} = 1) \times *
\end{equation}
where $*$ is the proportion of tokens in document $d$ assigned to topics associated with cluster $c$.

Given that we do not directly observe the generative process, the object of inference becomes the posterior distribution of our model parameters given the data. This problem is analytically intractable, so we must approximate the posterior distribution via Markov chain Monte Carlo methods. We perform inference for this model via block Metropolis within Gibbs sampling\footnote{The inference algorithm is currently implemented in a beta version as an R package, and is available here: \href{https://github.com/matthewjdenny/ContentStructure}{github.com/matthewjdenny/ContentStructure}}. A further discussion of this model will be provided in a related paper, but this model provides a powerful and flexible framework to investigate the content-conditional gendered patterns of communication in an organization. 

\subsection{Inference}
 We have to invert the generative process to perform inference on the model parameters.
We use block Metropolis Hastings within Gibbs sampling.
A beta R package is available for those interested.
Discuss our model specification and justify our hyper-parameter choices.


\section{Gender Mixing}


 \begin{figure}
\caption{\label{fig:Mixing parameters} Mixing Parameters}	
\centering
\includegraphics[width = 0.48\textwidth]{images/Mixing_Parameter_Plot.pdf}
\end{figure}

Since we have modeled content and structure together, we have broken up the confounding we were concerned about, and we can now interpret the gender mixing parameters that come out of our model with more confidence.
	
Lets return to the question we posed at the end of our descriptive analysis section: is there gender bias in the patterns of communication in these county governments. 
	
 Add in Bruce's MV KS test results for each county here.
	
Interpret the results.
	
	
Take as a case study, a county with the highest degree of gender bias, as identified by this method.
	
interpret the topic model output from two different clusters and discuss
	
	







\begin{acknowledgments}
This work was supported by US National Science Foundation Grant CISE-1320219 (Hanna Wallach and Bruce A. Desmarais, PIs)
\vspace{-.5cm}
\end{acknowledgments}
\bibliographystyle{plain}
\bibliography{PINLab.bib}


\end{article}



\end{document}


