\documentclass[10pt,english,oneside]{article}
\usepackage[pdftex]{graphics}
\usepackage{amsmath,amssymb,rotating,multirow}
\usepackage{palatino}
\usepackage{url}

\usepackage{subfigure}

\usepackage{natbib}

\usepackage[plain]{algorithm}
\usepackage{algpseudocode}

\usepackage{parskip}
\usepackage{setspace}
\onehalfspacing

\usepackage{geometry}
\geometry{verbose,letterpaper}

\setlength{\parskip}{10pt}

\title{Our Model}
\author{}
\date{\today}

\bibliographystyle{apalike}

\newcommand{\comment}[1]{}

% bold letters and symbols

\newcommand{\ba}{\boldsymbol{a}}
\newcommand{\bb}{\boldsymbol{b}}
\newcommand{\bc}{\boldsymbol{c}}
\newcommand{\bm}{\boldsymbol{m}}
\newcommand{\bn}{\boldsymbol{n}}
\newcommand{\bs}{\boldsymbol{s}}
\newcommand{\bt}{\boldsymbol{t}}
\newcommand{\bw}{\boldsymbol{w}}
\newcommand{\bx}{\boldsymbol{x}}
\newcommand{\by}{\boldsymbol{y}}
\newcommand{\bz}{\boldsymbol{z}}

\newcommand{\bN}{\boldsymbol{N}}

\newcommand{\bzero}{\boldsymbol{0}}

\newcommand{\bchi}{\boldsymbol{\chi}}
\newcommand{\bgamma}{\boldsymbol{\gamma}}
\newcommand{\bphi}{\boldsymbol{\phi}}
\newcommand{\btheta}{\boldsymbol{\theta}}

\newcommand{\I}{\boldsymbol{\text{I}}}

% distributions

\newcommand{\Bern}{\text{Bern}}
\newcommand{\Dir}{\text{Dir}}
\newcommand{\Mult}{\text{Mult}}
\newcommand{\Pois}{\text{Pois}}
\newcommand{\Norm}{\mathcal{N}}
\newcommand{\Unif}{\text{Unif}}

\newcommand{\g}{\,|\,}
\newcommand{\teq}{\!=\!}
\newcommand{\is}{:=}

\newcommand{\intd}{\,\textrm{d}}

\renewcommand{\algorithmiccomment}[1]{\% #1}

\newcommand{\argmax}{\operatornamewithlimits{argmax}}


\begin{document}

First we generate the global (corpus-wide) variables. There are two
main sets of global variables: those that describe the topics people
talk about and those that describe how people interact (interaction
patterns). These variables are linked by a third set of variables that
associate each topic with the pattern that best describes how people
interact when talking about that topic.

There are $T$ topics. Each topic $\bphi^{(t)}$ is a discrete
distribution over $V$ word types.

\begin{algorithm}[h]
  \begin{algorithmic}[1]
    \For{$t = 1$ to $T$}
    \State draw $\bphi^{(t)} \sim \Dir(\beta, \bm)$
    \EndFor
  \end{algorithmic}
\end{algorithm}

There are $C$ interaction patterns. Each interaction pattern consists
of an intercept $b^{(c)} \in \mathbb{R}$, a coefficient vector
$\bgamma^{(c)} \in \mathbb{R}^P$, and a set of $A$ positions $\{
\bs^{(c)}_a \in \mathbb{R}^K \}_{a=1}^A$---one for each person.  We
also associate each sender--recipient pair with an observed
$P$-dimensional vector of covariates $\bx^{(ar)}$; however, we assume
that our generative process is conditioned on these covariates.

\begin{algorithm}[h]
  \begin{algorithmic}[1]
    \For{$c = 1$ to $C$}
    \State draw $b^{(c)} \sim \Norm(\mu, \sigma_1^2)$
    \State draw $\bgamma^{(c)} \sim \Norm(\bzero, \sigma_2^2 \, \I_P)$
    \For{$a = 1$ to $A$}
    \State draw $\bs^{(c)}_a \sim \Norm(\bzero, \sigma_2^2\, \I_K)$
    \EndFor
    \For{$a=1$ to $A$}
    \For{$r=1$ to $A$}
    \If{$r \neq a$}
    \State set $p^{(c)}_{ar} = \sigma(b^{(c)} + {\bgamma^{(c)}}^{\top}
    \bx^{(ar)} - ||\bs_{a}^{(c)} - \bs_{r}^{(c)}||)$
    \Else
    \State set $p^{(c)}_{ar} = 0$
    \EndIf
    \EndFor
    \EndFor
    \EndFor
  \end{algorithmic}
\end{algorithm}

The topics and interaction patterns are tied together via a set of $T$
categorical variables (one per topic). These variables associate each
topic with a single interaction pattern.

\begin{algorithm}[h]
  \begin{algorithmic}[1]
    \For{$t = 1$ to $T$}
    \State draw $l_t \sim \Unif(1, C)$
    \EndFor
  \end{algorithmic}
\end{algorithm}

\pagebreak

Then, we generate the local variables. There are $D$ emails. We assume
that each email's sender $a^{(d)} \in [A]$ and length $N^{(d)} \in
\mathbb{N}_0$ are observed; we do not generate these variables.

\begin{algorithm}[h]
  \begin{algorithmic}[1]
    \For{$d=1$ to $D$}
    \State draw $\btheta^{(d)} \sim \Dir(\alpha, \bm)$
    \State set $\bar{N}^{(d)} = \textrm{max}(1, N^{(d)})$
    \For{$n=1$ to $\bar{N}^{(d)}$}    
    \State draw $z_n^{(d)} \sim \btheta^{(d)}$
    \If{$N^{(d)} \neq 0$}
    \State draw $w_n^{(d)} \sim \bphi^{(z_n^{(d)})}$
    \EndIf
    \EndFor
    \For{$t=1$ to $T$}
    \State set $\bar{N}^{(t|d)} = \sum_{n=1}^{\bar{N}^{(d)}}
    \delta(z_n^{(d)} \teq t)$
    \EndFor
    \For{$r=1$ to $A$}
    \State draw $y_r^{(d)} \sim \Bern(    \sum_{t=1}^T \frac{\bar{N}^{(t|d)}}{\bar{N}^{(d)}}
    p^{(l_t)}_{a^{(d)}r})$
    \EndFor
    \EndFor
  \end{algorithmic}
\end{algorithm}

This generative process implies a particular factorization of the
joint distribution over $\Phi = \{ \bphi^{(t)} \}_{t=1}^T$,
$\mathcal{B} = \{ b^{(c)} \}_{c=1}^C$, $\Gamma = \{ \bgamma^{(c)}
\}_{c=1}^C$, $\mathcal{S} = \{ \{ \bs^{(c)}_a \} \}_{c=1}^C$,
$\mathcal{L} = \{ l_t\}_{t=1}^T$, $\Theta = \{ \btheta^{(d)}
\}_{d=1}^D$, $\mathcal{Z} = \{ \bz^{(d)} \}_{d=1}^D$, $\mathcal{W} =
\{ \bw^{(d)} \}_{d=1}^D$, and $\mathcal{Y} = \{ \by^{(d)} \}_{d=1}^D$
given $\mathcal{X} = \{ \{
\bx^{(ar)} \}_{r=1}^A \}_{a=1}^A$ and 
$\mathcal{A} = \{ a^{(d)} \}_{d=1}^D$:
\begin{align}
  &P(\Phi, \mathcal{B}, \Gamma, \mathcal{S}, \mathcal{L}, \Theta,
  \mathcal{Z}, \mathcal{W}, \mathcal{Y} \g \mathcal{X}, \mathcal{A})
  \notag \\
  &\qquad = P(\Phi) P(\mathcal{B}) P(\Gamma) P(\mathcal{S}) P(\mathcal{L})
  P(\Theta) P(\mathcal{Z} \g \Theta) P(\mathcal{W} \g \mathcal{Z},
  \Phi) P(\mathcal{Y} \g \mathcal{B}, \Gamma, \mathcal{S},
  \mathcal{L}, \mathcal{Z}, \mathcal{X}, \mathcal{A}).
  \end{align}
We can simplify this further by integrating out $\Phi$ and $\Theta$
using Dirichlet--multinomial conjugacy:
\begin{equation}
  P(\mathcal{B}, \Gamma, \mathcal{S}, \mathcal{L},
  \mathcal{Z}, \mathcal{W}, \mathcal{Y} \g \mathcal{X}, \mathcal{A}) =
  P(\mathcal{B}) P(\Gamma) P(\mathcal{S}) P(\mathcal{L})
  P(\mathcal{Z})
  P(\mathcal{W} \g \mathcal{Z}) P(\mathcal{Y} \g \mathcal{B}, \Gamma,
  \mathcal{S},
    \mathcal{L}, \mathcal{Z}, \mathcal{X}, \mathcal{A}).
  \end{equation}

Our inference goal is to draw samples from the posterior distribution
\begin{equation}
  P(\mathcal{B}, \Gamma, \mathcal{S}, \mathcal{L},
  \mathcal{Z} \g \mathcal{W}, \mathcal{Y}, \mathcal{X}, \mathcal{A})
  \propto P(\mathcal{B}, \Gamma, \mathcal{S}, \mathcal{L},
    \mathcal{Z}, \mathcal{W}, \mathcal{Y} \g \mathcal{X}, \mathcal{A}).
  \end{equation}
In practice, we can achieve this inference goal by sequentially
resampling the value of each latent variable (i.e., $b^{(c)}$,
$\bgamma^{(c)}$, $\bs^{(c)}_a$, $l_t$, or $z^{(d)}_n$) from its
conditional posterior distribution.

Let's first consider $z_n^{(d)}$, whose conditional posterior probability of being
topic $t$ is
\begin{align}
  &P(z_n^{(d)} \teq t \g \mathcal{B}, \Gamma, \mathcal{S}, \mathcal{L},
    \mathcal{Z}_{\setminus d, n}, \mathcal{W}, \mathcal{Y},
    \mathcal{X}, \mathcal{A}) \notag \\
    &\qquad \propto P(z_n^{(d)} \teq t, w_n^{(d)}, \by^{(d)} \g
    \mathcal{B}, \Gamma, \mathcal{S},
    \mathcal{L},\mathcal{Z}_{\setminus d, n}, \mathcal{W}_{\setminus
      d, n},
    \mathcal{Y}_{\setminus d},
        \mathcal{X}, \mathcal{A}) \\
        &\qquad = P(z_n^{(d)} \teq t \g \mathcal{Z}_{\setminus d, n})
        P(w_n^{(d)} \g z_n^{(d)} \teq t, \mathcal{W}_{\setminus d, n},
        \mathcal{Z}_{\setminus d, n})
        P(\by^{(d)} \g \mathcal{B},
        \Gamma, \mathcal{S}, \mathcal{L}, z_n^{(d)} \teq t,
        \mathcal{Z}_{\setminus d, n}, \mathcal{X}, \mathcal{A}).
\end{align}
We have dropped the conditional dependence on
$\mathcal{Y}_{\setminus d}$ because $\by^{(d)} \perp
\mathcal{Y}_{\setminus d}$. We know that
\begin{equation}
  P(z_n^{(d)} \teq t \g \mathcal{Z}_{\setminus d, n}) = \frac{P(z_n^{(d)} \teq t, \mathcal{Z}_{\setminus d,
      n})}{P(\mathcal{Z}_{\setminus d, n})} = \frac{\bar{N}^{(t|d)}_{\setminus d, n} +
    \frac{\alpha}{T}}{\bar{N}^{(d)} - 1 + \alpha}.
\end{equation}
(The
derivation is the same as for LDA.) We
also know that 
\begin{equation}
  P(w_n^{(d)} \g z_n^{(d)} \teq t, \mathcal{W}_{\setminus d, n},
  \mathcal{Z}_{\setminus d, n}) = \frac{P(w^{(d)}_n,
    \mathcal{W}_{\setminus d, n} \g z_n^{(d)} \teq t,
    \mathcal{Z}_{\setminus d, n})}{P(\mathcal{W}_{\setminus d, n} \g
    \mathcal{Z}_{\setminus d, n})} =
  \frac{N^{(w_n^{(d)}|t)}_{\setminus d, n} +
    \frac{\beta}{V}}{N^{(t)}_{\setminus d, n} + \beta},
\end{equation}
where $N^{(w_n^{(d)}|t)}_{\setminus d, n}$ is the number of tokens
assigned to topic $t$ whose type is the same as that of $w_n^{(d)}$,
excluding $w_n^{(d)}$ itself, and $N^{(t)}_{\setminus d, n} =
\sum_{v=1}^V N^{(v|t)}_{\setminus d, n}$. Finally, we know that
\begin{align}
  &P(\by^{(d)} \g \mathcal{B}, \Gamma, \mathcal{S}, \mathcal{L},
  z_n^{(d)} \teq t, \mathcal{Z}_{\setminus d, n}, \mathcal{X},
  \mathcal{A}) \notag \\
  &\qquad  =
  \prod_{r=1}^A 
  \left(   \sum_{t'=1}^T \frac{\bar{N}^{(t'|d)}_{\setminus d, n} +
    \delta(t' \teq t)}{\bar{N}^{(d)}} p_{a^{(d)}r}^{(l_{t'})} \right)^{y^{(d)}_r} \left( 1 -   \sum_{t'=1}^T \frac{\bar{N}^{(t'|d)}_{\setminus d, n} +
    \delta(t' \teq t)}{\bar{N}^{(d)}} p_{a^{(d)}r}^{(l_{t'})} \right)^{1 - y^{(d)}_r}.
  \end{align}.
Therefore, if $N^{(d)} > 0$, then 
\begin{align}
  &P(z_n^{(d)} \teq t \g \mathcal{B}, \Gamma, \mathcal{S}, \mathcal{L},
    \mathcal{Z}_{\setminus d, n}, \mathcal{W}, \mathcal{Y},
    \mathcal{X}, \mathcal{A}) \notag \\
        &\qquad \propto \left( \bar{N}^{(t|d)}_{\setminus d, n} +
    \frac{\alpha}{T}\right) \frac{N^{(w_n^{(d)}|t)}_{\setminus
        d, n} + \frac{\beta}{V}}{N^{(t)}_{\setminus d, n} + \beta}
    \times {} \notag \\
    &\qquad \quad
  \prod_{r=1}^A 
  \left(   \sum_{t'=1}^T \frac{\bar{N}^{(t'|d)}_{\setminus d, n} +
    \delta(t' \teq t)}{\bar{N}^{(d)}} p_{a^{(d)}r}^{(l_{t'})} \right)^{y^{(d)}_r} \left( 1 -   \sum_{t'=1}^T \frac{\bar{N}^{(t'|d)}_{\setminus d, n} +
    \delta(t' \teq t)}{\bar{N}^{(d)}} p_{a^{(d)}r}^{(l_{t'})} \right)^{1 - y^{(d)}_r},
\end{align}
If $N^{(d)} \teq 0$, the first term becomes $\frac{\alpha}{T}$ and
disappears because it is a constant. The second term disappears
because there are no tokens. Since $\bar{N}^{(t'|d)}_{\setminus
  d, n} \teq 0 $, we therefore have
\begin{equation}
  P(z_1^{(d)} \teq t \g \mathcal{B}, \Gamma, \mathcal{S}, \mathcal{L},
    \mathcal{Z}_{\setminus d, n}, \mathcal{W}, \mathcal{Y},
    \mathcal{X}, \mathcal{A}) 
    \propto 
    \prod_{r=1}^A \left( p_{a^{(d)}r}^{(l_t)} \right)^{y^{(d)}_r} \left( 1 - p_{a^{(d)}r}^{(l_t)}\right)^{1 - y^{(d)}_r}.
\end{equation}

Next, let's consider $l_t$, whose conditional posterior probability
of being latent space $c$
\begin{align}
  &P(l_t \teq c \g \mathcal{B}, \Gamma, \mathcal{S},
  \mathcal{L}_{\setminus t}, \mathcal{Z}, \mathcal{W}, \mathcal{Y},
  \mathcal{X}, \mathcal{A}) \notag \\
  &\qquad \propto P(l_t, \mathcal{Y} \g \mathcal{B},
  \Gamma, \mathcal{S},
  \mathcal{L}_{\setminus t}, \mathcal{Z}, \mathcal{W},
  \mathcal{X}, \mathcal{A})\\
  &\qquad = P(l_t \teq c) P(\mathcal{Y} \g \mathcal{B}, \Gamma,
  \mathcal{S}, l_t \teq c, \mathcal{L}_{\setminus t}, \mathcal{Z},
  \mathcal{X}, \mathcal{A}).
\end{align}
We know that $P(l_t \teq c) = \frac{1}{C}$. Since this is a constant,
this term disappears. We also know that
\begin{align}
  &P(\mathcal{Y} \g \mathcal{B}, \Gamma, \mathcal{S}, l_t \teq c,
  \mathcal{L}_{\setminus t}, \mathcal{Z}, \mathcal{X}, \mathcal{A})
  \notag \\
  &\qquad =
  \prod_{d=1}^D \prod_{r=1}^A
  \left(   \sum_{t'=1}^T \frac{\bar{N}^{(t'|d)}}{\bar{N}^{(d)}} p_{a^{(d)}r}^{(l_{t'})}
  \right)^{y^{(d)}_r} \left( 1 -   \sum_{t'=1}^T
  \frac{\bar{N}^{(t'|d)}}{\bar{N}^{(d)}} p_{a^{(d)}r}^{(l_{t'})}\right)^{1 - y^{(d)}_r},
\end{align}
with $l_t \teq c$ throughout. (Note that in the product over $d$, I
think we need only consider those emails that actually use topic $t$;
the others will have no terms involving $l_t$.)  Therefore,
\begin{align}
  &P(l_t \teq c \g \mathcal{B}, \Gamma, \mathcal{S},
  \mathcal{L}_{\setminus t}, \mathcal{Z}, \mathcal{W}, \mathcal{Y},
  \mathcal{X}, \mathcal{A}) \notag\\
  &\qquad \propto \prod_{d=1}^D \prod_{r=1}^A
  \left(   \sum_{t'=1}^T \frac{\bar{N}^{(t'|d)}}{\bar{N}^{(d)}} p_{a^{(d)}r}^{(l_{t'})}
  \right)^{y^{(d)}_r} \left( 1 -   \sum_{t'=1}^T
  \frac{\bar{N}^{(t'|d)}}{\bar{N}^{(d)}} p_{a^{(d)}r}^{(l_{t'})}\right)^{1 - y^{(d)}_r}.
\end{align}

We can resample the values of $b^{(c)}$, $\bgamma^{(c)}$, and
$\bs^{(c)}_a$ using Metropolis--Hastings or slice sampling. With
either algorithm, we can separately resample the values of the
variables or we can jointly resample the values of $\mathcal{B}$,
$\Gamma$, and $\mathcal{S}$. Let's first consider the latter scenario
using Metropolis--Hastings.

As a quick reminder, Metropolis--Hastings uses a proposal density $Q$,
which depends on the current values of the variables to be sampled:
$Q(\mathcal{B}', \Gamma', \mathcal{S}' \g \mathcal{B}, \Gamma,
\mathcal{S})$. Here, we'll assume that $Q$ is a multivariate Gaussian
distribution, with a diagonal covariance matrix, centered on a vector
that consists of the current state of the Markov chain (i.e., the
current values of $\mathcal{B}$, $\Gamma$, and $\mathcal{S}$). Each
iteration of the algorithm involves drawing a proposal $\mathcal{B}'$,
$\Gamma'$, and $\mathcal{S}'$ from $Q$ given the current state of the
Markov chain $\mathcal{B}$, $\Gamma$, and $\mathcal{S}$. This proposal
is accepted as the new state of the chain if
\begin{equation}
  \frac{Q(\mathcal{B}, \Gamma, \mathcal{S} \g \mathcal{B}', \Gamma',
    \mathcal{S}')}{Q(\mathcal{B}', \Gamma', \mathcal{S}' \g \mathcal{B},
    \Gamma, \mathcal{S})}
  \frac{P(\mathcal{B}', \Gamma',
    \mathcal{S}' \g \mathcal{L}, \mathcal{Z}, \mathcal{W},
    \mathcal{Y}, \mathcal{X}, \mathcal{A})}{
        P(\mathcal{B}, \Gamma,
    \mathcal{S} \g \mathcal{L}, \mathcal{Z}, \mathcal{W},
    \mathcal{Y}, \mathcal{X}, \mathcal{A})} \geq 1.
\end{equation}
If $Q$ is a Gaussian distribution, then the first term disappears because
\begin{equation}
\frac{\frac{1}{\sqrt{2 \pi \sigma^2}} \exp{\left( - \frac{(x -
      x')^2}{2 \sigma^2} \right)}}{\frac{1}{\sqrt{2 \pi \sigma^2}}
  \exp{\left( - \frac{(x' - x)^2}{2 \sigma^2} \right)}} = \exp{\left(
  \frac{1}{2 \sigma^2} \left( (x' - x)^2 - (x - x')^2\right)\right)} = \exp{(0)} = 1.
\end{equation}
To simplify the second term, note that
\begin{equation}
  \frac{P(\mathcal{B}', \Gamma',
    \mathcal{S}' \g \mathcal{L}, \mathcal{Z}, \mathcal{W},
    \mathcal{Y}, \mathcal{X}, \mathcal{A})}{
        P(\mathcal{B}, \Gamma,
    \mathcal{S} \g \mathcal{L}, \mathcal{Z}, \mathcal{W},
    \mathcal{Y}, \mathcal{X}, \mathcal{A})} =
  \frac{P(\mathcal{B}', \Gamma',
    \mathcal{S}', \mathcal{L}, \mathcal{Z}, \mathcal{W},
    \mathcal{Y} \g \mathcal{X}, \mathcal{A})}{
        P(\mathcal{B}, \Gamma,
    \mathcal{S}, \mathcal{L}, \mathcal{Z}, \mathcal{W},
    \mathcal{Y} \g \mathcal{X}, \mathcal{A})}  
  \end{equation}
because the normalization constants that turn the joint distribution
into the conditional distribution cancel. Any terms in the joint
distribution that don't involve $\mathcal{B}$, $\Gamma$, or
$\mathcal{S}$ also cancel, so
\begin{align}
  &\frac{
    P(\mathcal{B}', \Gamma', \mathcal{S}', \mathcal{L},
  \mathcal{Z}, \mathcal{W}, \mathcal{Y} \g \mathcal{X}, \mathcal{A})
  }{    P(\mathcal{B}, \Gamma, \mathcal{S}, \mathcal{L},
  \mathcal{Z}, \mathcal{W}, \mathcal{Y} \g \mathcal{X}, \mathcal{A})}
  \notag \\
  &\qquad =
  \frac{
  P(\mathcal{B'}) P(\Gamma') P(\mathcal{S}') P(\mathcal{L})
  P(\mathcal{Z})
  P(\mathcal{W} \g \mathcal{Z}) P(\mathcal{Y} \g \mathcal{B}',
  \Gamma', \mathcal{S}', \mathcal{L}, \mathcal{Z}, \mathcal{X}, \mathcal{A})}{
  P(\mathcal{B}) P(\Gamma) P(\mathcal{S}) P(\mathcal{L})
  P(\mathcal{Z})
  P(\mathcal{W} \g \mathcal{Z}) P(\mathcal{Y} \g \mathcal{B}, \Gamma,
  \mathcal{S}, \mathcal{L}, \mathcal{Z},
  \mathcal{X}, \mathcal{A})}\\
  &\qquad = \frac{P(\mathcal{B}')P(\Gamma')P(\mathcal{S'})P(\mathcal{Y} \g
    \mathcal{B}', \Gamma', \mathcal{S}', \mathcal{L}, \mathcal{Z},
    \mathcal{X}, \mathcal{A})}{P(\mathcal{B}) P(\Gamma) P(\mathcal{S})
  P(\mathcal{Y} \g \mathcal{B}, \Gamma, \mathcal{S}, \mathcal{L},
  \mathcal{Z}, \mathcal{X}, \mathcal{A})}\\
  &\qquad = \frac{P(\mathcal{B}') P(\Gamma') P(\mathcal{S}')}{P(\mathcal{B})
    P(\Gamma) P(\mathcal{S})} \frac{
    \prod_{d=1}^D \prod_{r=1}^A \left(
    \sum_{t=1}^T \frac{\bar{N}^{(t|d)}}{\bar{N}^{(d)}}
    {p'}^{(l_t)}_{a^{(d)}r}
  \right)^{y^{(d)}_r} \left( 1 -
    \sum_{t=1}^T \frac{\bar{N}^{(t|d)}}{\bar{N}^{(d)}}
    {p'}^{(l_t)}_{a^{(d)}r}  
  \right)^{1 - y^{(d)}_r}}{
    \prod_{d=1}^D \prod_{r=1}^A \left(
    \sum_{t=1}^T \frac{\bar{N}^{(t|d)}}{\bar{N}^{(d)}}
    {p}^{(l_t)}_{a^{(d)}r}
  \right)^{y^{(d)}_r} \left( 1 -
    \sum_{t=1}^T \frac{\bar{N}^{(t|d)}}{\bar{N}^{(d)}}
    {p}^{(l_t)}_{a^{(d)}r}  
  \right)^{1 - y^{(d)}_r}}    \\
  &\qquad = \frac{P(\mathcal{B}') P(\Gamma') P(\mathcal{S}')}{P(\mathcal{B})
    P(\Gamma) P(\mathcal{S})} \prod_{d=1}^D \prod_{r=1}^A
  \left( \frac{    \sum_{t=1}^T \frac{\bar{N}^{(t|d)}}{\bar{N}^{(d)}}
    {p'}^{(l_t)}_{a^{(d)}r}
}{          \sum_{t=1}^T \frac{
      \bar{N}^{(t|d)}}{\bar{N}^{(d)}}
    {p}^{(l_t)}_{a^{(d)}r}} \right)^{y^{(d)}_r}
  \left( \frac{  1 -
    \sum_{t=1}^T \frac{\bar{N}^{(t|d)}}{\bar{N}^{(d)}}
    {p'}^{(l_t)}_{a^{(d)}r}  }{  1 -
    \sum_{t=1}^T \frac{\bar{N}^{(t|d)}}{\bar{N}^{(d)}}
    {p}^{(l_t)}_{a^{(d)}r}  
} \right)^{1 - y^{(d)}_r},
\end{align}
where ${p'}^{(c)}_{ar} = \sigma({b'}^{(c)} + {{\bgamma'}^{(c)}}^{\top}
\bx^{(ar)} - ||{\bs'}^{(c)}_a - {\bs'}^{(c)}_r||)$.

Bruce suggests an alternative to step 14 of the generative process for
the local variables:
\begin{equation}
  y^{(d)}_r \sim \Bern \left(
  \frac{
    \prod_{c=1}^C
  \left(p^{(c)}_{a^{(d)}r}\right)^{\bar{N}^{(c|d)}}}{
    \prod_{c=1}^C
  \left(p^{(c)}_{a^{(d)}r}\right)^{\bar{N}^{(c|d)}} + \prod_{c=1}^C
  \left(1 - p^{(c)}_{a^{(d)}r}\right)^{\bar{N}^{(c|d)}}}  
  \right),
  \end{equation}
where $\bar{N}^{(c|d)} = \sum_{t=1}^T \bar{N}^{(t|d)} \delta(l_t \teq
c)$. Currently, however, we have 
\begin{align}
y_r^{(d)} \sim \Bern \left(
\sum_{t=1}^T \frac{\bar{N}^{(t|d)}}{\bar{N}^{(d)}}
    p^{(l_t)}_{a^{(d)}r}
    \right) = \Bern \left(
\sum_{c=1}^C \frac{\bar{N}^{(c|d)}}{\bar{N}^{(d)}}
    p^{(c)}_{a^{(d)}r}    
    \right),
\end{align}
which implies that $y^{(d)}_r$ is drawn from a mixture model, where
each mixture component is an interaction pattern. The (normalized)
weight for each interaction pattern $c$ is the proportion of tokens in
email $d$ that are associated with (topics associated with) that
interaction pattern. In contrast, Bruce's suggestion implies that
$y^{(d)}$ is drawn from a product model. Here too, the components are
interaction patterns, however the weight for each pattern is now the
number of tokens in email $d$ that are associated with that pattern,
and the components' contributions are combined differently.

In general, a $C$-component mixture model for a random variable $y \in
\{0, 1\}$ implies that $y$ is drawn from a linear combination of $C$
Bernoulli components, parameterized by $p^{(1)}, \ldots,
p^{(C)}$---i.e., 
\begin{align}
  P(y \teq 1) = p
  &= \frac{\sum_{c=1}^C \pi^{(c)}
    p^{(c)}}{\sum_{c=1}^C \pi^{(c)} p^{(c)} + \sum_{c=1} \pi^{(c)}
    \left( 1 - p^{(c)}\right)}\\
  &= \frac{\sum_{c=1}^C \pi^{(c)}
    p^{(c)}}{\sum_{c=1}^C \pi^{(c)} \left( p^{(c)} + 1 -
    p^{(c)}\right)}\\
  \label{eqn:mixture}
  &= \sum_{c=1}^C \frac{\pi^{(c)}}{\sum_{c=1}^C \pi^{(c)}} p^{(c)},
\end{align}
where $\pi^{(1)} \geq 0, \ldots, \pi^{(C)} \geq 0$ are the component
weights. In contrast, a $C$-component product model for $y$, formed
from the same components and component weights, implies that
\begin{equation}
  P(y \teq 1) = p = \frac{\prod_{c=1}^C \left( p^{(c)}
    \right)^{\pi^{(c)}}}{\prod_{c=1}^C \left(
    p^{(c)}\right)^{\pi^{(c)}}+ \prod_{c=1}^C \left( 1 -
    p^{(c)}\right)^{\pi^{(c)}}}.
\end{equation}
  
There are two main differences between a mixture model and a product
model. The first is in the way that the components' contributions are
combined. In a mixture model, the component-specific probabilities are
combined additively (i.e., via an ``or'' function); in a product
model, the component-specific probabilities are combined
multiplicatively (i.e., via an ``and'' function).

To illustrate this difference, let's consider a simple example in
which the value of $y \in \{0, 1\}$ is drawn from either a mixture
model with two Bernoulli components, parameterized by $p^{(1)}$ and
$p^{(2)}$, respectively, or a product model formed from the same two
components. In both models, we'll assume that the components are
equally weighted with $\pi^{(1)} = \pi^{(2)} = 1$. The mixture model
therefore implies that
\begin{equation}
  P(y \teq 1) = p = \frac{1}{2} \sum_{c=1}^2 p^{(c)},
  \end{equation}
while the product model implies that
\begin{equation}
P(y \teq 1) = p = \frac{\prod_{c=1}^2 p^{(c)}}{\prod_{c=1}^2 p^{(c)} +
  \prod_{c=1}^2 (1-p^{(c)})}.
\end{equation}

In the mixture model, the components can compensate for one
another. For example, if $p^{(1)} \teq 0.9$ and $p^{(2)} \teq 0.2$,
then $p = 0.55$. To better understand this example, let's quantify the
uncertainty of a distribution as its entropy, so less certain
distributions have higher entropy and vice versa. For a two
dimensional distribution $q$, $0 \leq H(q) \leq 1.0$. Here, $H\left(
p^{(1)} \right) \teq 0.469$, $H\left( p^{(2)} \right) \teq 0.722$, and
$H(p) \teq 0.993$. Because the components disagree, both components
are made less certain by their combination. In contrast, if $p^{(1)}
\teq 0.9$ and $p^{(2)} \teq 0.7$, then $p = 0.8$. Here, $H\left(
p^{(1)} \right) \teq 0.469$, $H\left( p^{(2)} \right) \teq 0.881$, and
$H(p) \teq 0.722$. Because the components agree, but with differing
certainties, the more certain component is made less certain while the
less certain component is made more so. This compensatory behavior has
the biggest substantive impact on components that are very
certain. For example, if $p^{(1)} \teq 0$ and $p^{(2)} \teq 0.4$, then
$p \teq 0.2$. Equivalently, $H\left(p^{(1)}\right) \teq 0$, $H\left(
p^{(2)} \right) \teq 0.971$, and $H(p) \teq 0.722$. In other words,
even though the first component was absolutely certain, the mixture
model is not certain (i.e., $p \neq 0)$ because of the second
component's non-zero probability.

In general, a mixture model cannot be more certain than its most
certain component. (It can, however, be more or less certain than the
other components, though.) We can use Jensen's inequality to see why
this is the case. Since entropy is a concave function, Jensen's
inequality implies that
\begin{equation}
  \label{eqn:jensen}
  H\left( \sum_{c=1}^C \frac{\pi^{(c)}}{\sum_{c=1}^C \pi^{(c)}}
  p^{(c)} \right) \geq \sum_{c=1}^C \frac{\pi^{(c)}}{\sum_{c=1}^C
    \pi^{(c)}} H\left(p^{(c)}\right).
\end{equation}
For a fixed set of components $p^{(1)}, \ldots, p^{(C)}$, making the
mixture model as certain as possible is equivalent to minimizing its
entropy with respect to the weights $\pi^{(1)}, \ldots,
\pi^{(C)}$---i.e.,
\begin{equation}
  \min_{\boldsymbol{\pi}} H\left( \sum_{c=1}^C
  \frac{\pi^{(c)}}{\sum_{c=1}^C \pi^{(c)}} p^{(c)} \right) \geq
  \min_{\boldsymbol{\pi}}\sum_{c=1}^C \frac{\pi^{(c)}}{\sum_{c=1}^C
    \pi^{(c)}} H\left(p^{(c)}\right).
\end{equation}
The right-hand side of equation~\ref{eqn:jensen} can only be minimized
by placing all the weight (i.e., $\sum_{c=1}^C \pi^{(c)}$) on the most
certain component (and no weight on the other components). Therefore,
\begin{equation}
  \label{eqn:mixture_certainty}
  \min_{\boldsymbol{\pi}} H\left( \sum_{c=1}^C
  \frac{\pi^{(c)}}{\sum_{c=1}^C \pi^{(c)}} p^{(c)} \right) \geq
  \min_{c} H\left(p^{(c)}\right).
\end{equation}
In other words, no matter how we choose the weights $\pi^{(1)},
\ldots, \pi^{(C)}$, the mixture model cannot be more certain than its
most certain component. Moreover, by examining
equation~\ref{eqn:mixture_certainty}, we can see that the mixture
model will only be as certain as its most certain component if
$p^{(1)} = \ldots = p^{(C)}$ or if we place all the weight on the most
certain component and none on the other components.

Returning to our model, let's consider an email whose tokens
$\bw^{(d)}$ are associated (via $\bz^{(d)}$ and $\mathcal{L}$) with
interaction patterns $c$ and $c'$. Even if interaction pattern $c$
specifies that sender $a^{(d)}$ never interacts with recipient $r$,
then provided interaction pattern $c'$ specifies a non-zero
probability of interaction, the probability of $a^{(d)}$ sending the
email to $r$ will be non-zero. For example, even if I never email my
boss about personal matters (but do email her about work), there is a
non-zero probability that I will send an her an email containing both
personal and work content. Is this a reasonable assumption?
Intuitively, I'm not sure that it is. Realistically, I can't see
myself sending my boss an email containing personal content, even if
some portion of the email is about work.

In contrast, a product model can be more or less certain than its most
certain component. For example, if $p^{(1)} \teq 0.9$ and $p^{(2)}
\teq 0.2$, then $p \teq 0.692$. Equivalently, $H\left( p^{(1)}\right)
\teq 0.469$, $H\left( p^{(2)}\right) \teq 0.722$, and $H(p) =
0.890$. In other words, both components are made less certain by their
combination. However, if $p^{(1)} \teq 0.9$ and $p^{(2)} \teq 0.7$,
then $p \teq 0.955$. Here, $H\left( p^{(1)} \right) \teq 0.469$, $H
\left( p^{(2)}\right) \teq 0.881$, and $H(p) \teq 0.267$. Both
components are made more certain by their combination---even the most
certain component. In addition to this difference, components can veto
one another. For example, if $p^{(1)} \teq 0$ and $p^{(2)} \teq 0.4$,
then $p \teq 0$. Equivalently, $H\left( p^{(1)} \right) \teq 0$, $H
\left( p^{(2)}\right) \teq 0.971$, and $H(p) \teq 0$. In other words,
the product model is certain that $y \teq 0$ because the first
component was certain.

Returning to Bruce's alternative to our model, if interaction pattern
$c$ specifies that sender $a^{(d)}$ never interacts with recipient
$r$, then the probability of $a^{(d)}$ sending to $r$ an email that
contains some tokens associated with interaction pattern $c$ is
zero. For example, if I never email my boss about personal matters
(but do email her about work), then I will not send her an email
containing both personal and work content. Is this a reasonable
assumption?  Intuitively, I think it is.

The second difference between a mixture model and a product model is
in the components' weights. Because the weights in a mixture model are
normalized (see equation~\ref{eqn:mixture}), their scale is
ignored. For example, in a two-component mixture model, $\pi^{(1)}
\teq 1$ and $\pi^{(2)} \teq 4$ and $\pi^{(1)} \teq 100$ and $\pi^{(2)}
\teq 400$ will both be normalized to $\pi^{(1)} \teq 0.2$ and
$\pi^{(2)} \teq 0.8$ and therefore yield the same mixture model.

In our model, when forming the probability of $y_r^{(d)} \teq 1$, the
weight for each interaction pattern $c$ is the number of tokens in
email $d$ that are associated with that interaction pattern. When
normalized, each weight becomes a proportion---i.e.,
$\frac{\pi^{(c)}}{\sum_{c=1}^C \pi^{(c)}} =
\frac{\bar{N}^{(c|d)}}{\bar{N}^{(d)}}$. Normalization means that two
emails with the same sender will have identical probabilities of being
sent to recipient $r$, provided their empirical distributions over
interaction patterns are the same. For example, let's consider two
such emails $d$ and $d'$ and assume that for every interaction
pattern, the second email contains ten times as many tokens associated
with that pattern as the first---i.e., $\bar{N}^{(c|d')} = 10 \cdot
\bar{N}^{(c|d)}$. Since
\begin{equation}
  \frac{\bar{N}^{(c|d')}}{\bar{N}^{(d')}} = \frac{\bar{N}^{(c|d')}}{\sum_{c=1}^C
    \bar{N}^{(c|d')}} = \frac{10 \cdot \bar{N}^{(c|d)}}{\sum_{c=1}^C 10 \cdot
    \bar{N}^{(c|d)}} = \frac{\bar{N}^{(c|d)}}{\sum_{c=1}^C \bar{N}^{(c|d)}} =
  \frac{\bar{N}^{(c|d)}}{\bar{N}^{(d)}},
\end{equation}
our model implies that
\begin{equation}
 P(y^{(d)}_r \teq 1) =
 \sum_{c=1}^C \frac{\bar{N}^{(c|d)}}{\bar{N}^{(d)}}
 p^{(c)}_{a^{(d)}r} =  \sum_{c=1}^C \frac{\bar{N}^{(c|d')}}{\bar{N}^{(d')}}
 p^{(c)}_{a^{(d')}r} = P(y^{(d')}_r \teq 1).
\end{equation}
In other words, our model treats the two emails identically, even
though they are different lengths, because their empirical
distributions over interaction patterns are the same. Equivalently,
our model assumes that email length does not affect the probability of
sending an email to a particular recipient; an empirical distribution
over interaction patterns formed from two tokens plays exactly the
same role as one formed from 200 tokens. Is this a reasonable
assumption? Intuitively, I'm not sure it is, given that the latter
distribution is formed from much more data than the former.

In the product model, the scale of the components' weights is not
ignored. For example, let's consider a two-component mixture model
with $p^{(1)} \teq 0.9$ and $p^{(2)} \teq 0.2$. If $\pi^{(1)} \teq
0.2$ and $\pi^{(2)} \teq 0.8$, then $p = 0.339$; however, if
$\pi^{(1)} \teq 1$ and $\pi^{(2)} \teq 4$, then $p = 0.033$. If
$\pi^{(1)} \teq 100$ and $\pi^{(2)} \teq 400$, then $p \teq 0$. In
other words, the larger the scale of the weights, the more certain the
product model is.

In Bruce's alternative model, when forming the probability of
$y_r^{(d)} \teq 1$, the weight for interaction pattern $c$ is the
number of tokens in email $d$ that are associated with that
interaction pattern---i.e., $\pi^{(c)} = \bar{N}^{(c|d)}$. Two emails
with the same sender will only have identical probabilities of being
sent to recipient $r$ if their empirical distributions over
interaction patterns \emph{and} their lengths are the same. Let's
again consider two such emails $d$ and $d'$ and assume that for every
interaction pattern, the second email contains ten times as many
tokens associated with that pattern as the first. Now,
\begin{align}
P(y^{(d')}_r \teq 1)
&= \frac{\prod_{c=1}^C \left( p^{(c)}_{a^{(d')}r}
  \right)^{\bar{N}^{(c|d')}}}{\prod_{c=1}^C \left(
  p^{(c)}_{a^{(d')}r}\right)^{\bar{N}^{(c|d')}} + \prod_{c=1}^C \left( 1 -
  p^{(c)}_{a^{(d')}r}\right)^{\bar{N}^{(c|d')}}}\\
&= \frac{\prod_{c=1}^C \left( p^{(c)}_{a^{(d)}r}
  \right)^{10 \cdot \bar{N}^{(c|d)}}}{\prod_{c=1}^C \left(
  p^{(c)}_{a^{(d)}r}\right)^{10 \cdot \bar{N}^{(c|d)}} + \prod_{c=1}^C \left( 1 -
  p^{(c)}_{a^{(d)}r}\right)^{10 \cdot \bar{N}^{(c|d)}}}\\
&= \frac{\left( \prod_{c=1}^C \left( p^{(c)}_{a^{(d)}r}
  \right)^{\bar{N}^{(c|d)}}\right)^{10}}{\left(\prod_{c=1}^C \left(
  p^{(c)}_{a^{(d)}r}\right)^{\bar{N}^{(c|d)}}\right)^{10} +\left( \prod_{c=1}^C \left( 1 -
  p^{(c)}_{a^{(d)}r}\right)^{\bar{N}^{(c|d)}}\right)^{10}}\\
&\neq P(y^{(d)}_r \teq 1).
  \end{align}
In other words, we can form $P(y^{(d')}_r \teq 1)$ by raising each of
the constituent terms in $P(y^{(d)}_r \teq 1)$ to a power of ten. This
power is effectively an ``inverse temperature''; larger values make
the distribution more certain. Therefore, in Bruce's alternative to
our model, if two emails $d$ and $d'$ have the same sender and
empirical distribution over interaction patterns but different
lengths, they will not be treated identically; if $N^{(d')} \geq
N^{(d)}$, then the distribution over $y^{(d')}_r$ will be more certain
than the distribution over $y^{(d)}_r$ for each recipient $r$. In
general, raising (unnormalized) probabilities to a power has a major
impact on the resultant distribution; the larger the power, the more
certain the distribution will be. (This principle is used in simulated
annealing; there, however, the powers are fractional to make
distributions less certain.) For example, if an email $d$ contains a
few hundred tokens, then its recipient distributions will be close to
delta spikes.

There are (at least) two ways to lessen the impact of email length and
preserve some uncertainty for longer emails. First, we could use
component weights that are proportions---i.e., $\pi^{(c)} =
\frac{\bar{N}(c|d)}{\bar{N}(d)}$. However, this would mean that email
length does not affect the probability of sending an email to a
particular recipient, just as in our model. As I argued earlier, in
the context of our model, I'm not sure this is a reasonable
assumption. Another, less drastic, option would be to use component
weights that are log counts---i.e., $\pi^{(c)} =
\log{(\bar{N}^{(c|d)})}$. Intuitively, this approach is appealing as
it implies that the length of an email does matter, but only in terms
of its order of magnitude.

How does moving to a product model affect our inference equations?

Let's first consider $z_n^{(d)}$. Under Bruce's alternative to our
model,
\begin{align}
  &P(z_n^{(d)} \teq t \g \mathcal{B}, \Gamma, \mathcal{S}, \mathcal{L},
  \mathcal{Z}_{\setminus d, n}, \mathcal{W}, \mathcal{Y}, \mathcal{X},
  \mathcal{A}) \notag \\
&\qquad \propto \left( \bar{N}^{(t|d)}_{\setminus d, n} +
      \frac{\alpha}{T}\right) \frac{N^{(w_n^{(d)}|t)}_{\setminus
                  d, n} + \frac{\beta}{V}}{N^{(t)}_{\setminus d, n} +
        \beta}
          \times {} \notag \\
&\qquad\quad \prod_{r=1}^A \left( \frac{
                \prod_{t'=1}^T \left(
                p^{(l_{t'})}_{a^{(d)}r}\right)^{\bar{N}^{(t'|d)}_{\setminus
                    d,n} + \delta(t' \teq t)}
              }{
                \prod_{t'=1}^T \left(
                p^{(l_{t'})}_{a^{(d)}r}\right)^{\bar{N}^{(t'|d)}_{\setminus
                    d,n} + \delta(t' \teq t)} +
                \prod_{t'=1}^T \left(1 - 
                p^{(l_{t'})}_{a^{(d)}r}\right)^{\bar{N}^{(t'|d)}_{\setminus
                    d,n} + \delta(t' \teq t)}                
              }\right)^{y^{(d)}_r} \times {} \notag \\
&\qquad \quad \left( \frac{
                \prod_{t'=1}^T \left(1 - 
                p^{(l_{t'})}_{a^{(d)}r}\right)^{\bar{N}^{(t'|d)}_{\setminus
                    d,n} + \delta(t' \teq t)}
              }{
                \prod_{t'=1}^T \left(
                p^{(l_{t'})}_{a^{(d)}r}\right)^{\bar{N}^{(t'|d)}_{\setminus
                    d,n} + \delta(t' \teq t)} +
                \prod_{t'=1}^T \left(1 - 
                p^{(l_{t'})}_{a^{(d)}r}\right)^{\bar{N}^{(t'|d)}_{\setminus
                    d,n} + \delta(t' \teq t)}                
              }\right)^{1 - y^{(d)}_r}.
\end{align}
If $N^{(d)} \teq 0$, the first term becomes $\frac{\alpha}{T}$ and
disappears because it is a constant. The second term disappears
because there are no tokens. Since $\bar{N}^{(t'|d)}_{\setminus d, n}
\teq 0$ and
$\frac{p^{(l_t)}_{a^{(d)}r}}{p^{(l_t)}_{a^{(d)}r} + \left(1 -
  p^{(l_t)}_{a^{(d)}r}\right)} = p^{(l_t)}_{a^{(d)}r}$,
we have
\begin{equation}
 P(z_1^{(d)} \teq t \g \mathcal{B}, \Gamma, \mathcal{S}, \mathcal{L},
  \mathcal{Z}_{\setminus d, n}, \mathcal{W}, \mathcal{Y}, \mathcal{X},
  \mathcal{A}) \propto \prod_{r=1}^A \left(
  p^{(l_t)}_{a^{(d)}r}
    \right)^{y^{(d)}_r}
  \left( 1 - p^{(l_t)}_{a^{(d)}r}\right)^{1 - y^{(d)}_r}.
\end{equation}

Next, let's consider $l_t$. Under Bruce's alternative to our model,
\begin{align}
&P(l_t \teq c \g \mathcal{B}, \Gamma, \mathcal{S},
  \mathcal{L}_{\setminus t}, \mathcal{Z}, \mathcal{W}, \mathcal{Y},
  \mathcal{X}, \mathcal{A}) \notag \\
  &\qquad \propto \prod_{d=1}^D \prod_{r=1}^A \left(
  \frac{\prod_{t'=1}^T
    \left( p^{(l_{t'})}_{a^{(d)}r}\right)^{\bar{N}^{(t'|d)}}
  }{\prod_{t'=1}^T     \left(
    p^{(l_{t'})}_{a^{(d)}r}\right)^{\bar{N}^{(t'|d)}} +
    \prod_{t'=1}^T     \left( 1 - p^{(l_{t'})}_{a^{(d)}r}\right)^{\bar{N}^{(t'|d)}}}
  \right)^{y^{(d)}_r} \times {} \notag \\
  &\qquad \quad \left(
  \frac{ \prod_{t'=1}^T
    \left( 1 - p^{(l_{t'})}_{a^{(d)}r}\right)^{\bar{N}^{(t'|d)}}
  }{
    \prod_{t'=1}^T     \left(
    p^{(l_{t'})}_{a^{(d)}r}\right)^{\bar{N}^{(t'|d)}}
     + \prod_{t'=1}^T
         \left( 1 -p^{(l_{t'})}_{a^{(d)}r}\right)^{\bar{N}^{(t'|d)}}
  }
  \right)^{1 - y^{(d)}_r},
\end{align}
with $l_t \teq c$ throughout. Note that in the product over $d$, I
think we need only consider those emails that actually use topic $t$;
the others will have no terms involving $l_t$. Also, if $N^{(d)} = 0$
then
\begin{equation}
  \frac{ \prod_{t'=1}^T
    \left(p^{(l_{t'})}_{a^{(d)}r}\right)^{\bar{N}^{(t'|d)}}
  }{
    \prod_{t'=1}^T     \left(
    p^{(l_{t'})}_{a^{(d)}r}\right)^{\bar{N}^{(t'|d)}}
     + \prod_{t'=1}^T
         \left( 1 -p^{(l_{t'})}_{a^{(d)}r}\right)^{\bar{N}^{(t'|d)}}
  } = p^{(l_{z_1^{(d)}})}_{y^{(d)}_r}.
  \end{equation}

We can resample the values of $b^{(c)}$, $\bgamma^{(c)}$, and
$\bs_a^{(c)}$ using Metropolis--Hastings or slice sampling. Let's
consider jointly resampling the values of $\mathcal{B}$, $\Gamma$, and
$\mathcal{S}$ using Metropolis--Hastings.

Under Bruce's alternative to our model, the Metropolis--Hastings
acceptance ratio is
\begin{align}
&\frac{P(\mathcal{B}', \Gamma', \mathcal{S}', \mathcal{L},
    \mathcal{Z}, \mathcal{W}, \mathcal{Y} \g \mathcal{X},
    \mathcal{A})}{P(\mathcal{B}, \Gamma, \mathcal{S}, \mathcal{L},
    \mathcal{Z}, \mathcal{W}, \mathcal{Y} \g \mathcal{X},
    \mathcal{A})}\notag \\
  &\qquad =
  \frac{P(\mathcal{B}')P(\Gamma')P(\mathcal{S}')}{P(\mathcal{B})P(\Gamma)
  P(\mathcal{S}')}\times{} \notag\\
  &\qquad \quad \prod_{d=1}^D \prod_{r=1}^A \left(
  \frac{
    \prod_{t=1}^T \left(
    p^{(l_t)}_{a^{(d)}r}\right)^{\bar{N}^{(t|d)}} + \prod_{t=1}^T
    \left( 1 - p^{(l_t)}_{a^{(d)}r}\right)^{\bar{N}^{(t|d)}}
  }{
    \prod_{t=1}^T \left(
    {p'}^{(l_t)}_{a^{(d)}r}\right)^{\bar{N}^{(t|d)}} + \prod_{t=1}^T
    \left( 1 - {p'}^{(l_t)}_{a^{(d)}r}\right)^{\bar{N}^{(t|d)}}
  }
  \prod_{t=1}^T \left( \frac{ {p'}^{(l_t)}_{a^{(d)}r}}{p^{(l_t)}_{a^{(d)}r}} \right)^{\bar{N}^{(t|d)}}  
  \right)^{y^{(d)}_r} \times {} \notag \\
  &\qquad \quad \left(
  \frac{
    \prod_{t=1}^T \left(
    p^{(l_t)}_{a^{(d)}r}\right)^{\bar{N}^{(t|d)}} + \prod_{t=1}^T
    \left( 1 - p^{(l_t)}_{a^{(d)}r}\right)^{\bar{N}^{(t|d)}}
  }{
    \prod_{t=1}^T \left(
    {p'}^{(l_t)}_{a^{(d)}r}\right)^{\bar{N}^{(t|d)}} + \prod_{t=1}^T
    \left( 1 - {p'}^{(l_t)}_{a^{(d)}r}\right)^{\bar{N}^{(t|d)}}
  }
  \prod_{t=1}^T \left( \frac{1 - {p'}^{(l_t)}_{a^{(d)}r}}{1 - p^{(l_t)}_{a^{(d)}r}} \right)^{\bar{N}^{(t|d)}}    
  \right)^{1 - y^{(d)}_r}\\
  &\qquad =   \frac{P(\mathcal{B}')P(\Gamma')P(\mathcal{S}')}{P(\mathcal{B})P(\Gamma)
  P(\mathcal{S}')}\times{} \notag\\  
  &\qquad \quad \prod_{d=1}^D \prod_{r=1}^A
  \frac{
    \prod_{t=1}^T \left(
    p^{(l_t)}_{a^{(d)}r}\right)^{\bar{N}^{(t|d)}} + \prod_{t=1}^T
    \left( 1 - p^{(l_t)}_{a^{(d)}r}\right)^{\bar{N}^{(t|d)}}
  }{
    \prod_{t=1}^T \left(
    {p'}^{(l_t)}_{a^{(d)}r}\right)^{\bar{N}^{(t|d)}} + \prod_{t=1}^T
    \left( 1 - {p'}^{(l_t)}_{a^{(d)}r}\right)^{\bar{N}^{(t|d)}}
  } \times {} \notag \\
&\qquad \quad   \left( \prod_{t=1}^T \left(\frac{{p'}^{(l_t)}_{a^{(d)}r}}{p^{(l_t)}_{a^{(d)}r}} \right)^{\bar{N}^{(t|d)}} \right)^{y^{(d)}_r}
  \left( \prod_{t=1}^T \left( \frac{1 - {p'}^{(l_t)}_{a^{(d)}r}}{1 - p^{(l_t)}_{a^{(d)}r}} \right)^{\bar{N}^{(t|d)}} \right)^{1 -y^{(d)}_r},
\end{align}
where ${p'}^{(c)}_{ar} = \sigma({b'}^{(c)} + {{\bgamma'}^{(c)}}^{\top}
\bx^{(ar)} - ||{\bs'}^{(c)}_a - {\bs'}^{(c)}_r||)$.

\end{document}
